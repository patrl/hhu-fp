% Created 2022-12-05 Mon 23:03
% Intended LaTeX compiler: pdflatex
\documentclass[letterpaper,parskip=half]{scrartcl}
\usepackage[utf8]{inputenc}
\usepackage[T1]{fontenc}
\usepackage{graphicx}
\usepackage{longtable}
\usepackage{wrapfig}
\usepackage{rotating}
\usepackage[normalem]{ulem}
\usepackage{amsmath}
\usepackage{amssymb}
\usepackage{capt-of}
\usepackage{hyperref}
\usepackage{braket}
\input{boilerplate}
\author{Patrick D. Elliott}
\date{\today}
\title{QR recap and argument raising}
\hypersetup{
 pdfauthor={Patrick D. Elliott},
 pdftitle={QR recap and argument raising},
 pdfkeywords={},
 pdfsubject={},
 pdfcreator={Emacs 28.2 (Org mode 9.5.5)}, 
 pdflang={English}}
\usepackage{biblatex}
\addbibresource{/home/patrl/repos/bibliography/master.bib}
\addbibresource{~/repos/bibliography/master.bib}
\begin{document}

\maketitle
\tableofcontents


\section{Exercises (from Coppock and Champollion)}
\label{sec:org4601281}

\subsection{Composing quantifiers in object position}
\label{sec:org1745c6e}

Produce a translation into the simply-typed lambda calculus for the following sentence:

\begin{exe}
\ex Beth speaks a European language.
\label{orga7cf4ec}
\end{exe}

Here are some translations to help you get started.  

\begin{itemize}
\item Beth \(\Rightarrow\) \(\mathbf{Beth} : E\)
\item speaks \(\Rightarrow\) \(\mathbf{speaks} : E \to E \to T\)
\item European \(\Rightarrow\) \(\lambda P_{ET }\,.\,\lambda x\,.\,\mathbf{european}(x) \wedge P(x) : (E \to T) \to E \to T\)
\item language \(\Rightarrow\) \(\mathbf{language} :E \to T\)
\item a \(\Rightarrow\) \(\lambda R_{ET}\,.\,\lambda S_{ET}\,.\,\exists x[P(x) \wedge Q(x)] : (E \to T) \to (E \to T) \to T\)
\end{itemize}

Some important facts about quantifier raising.
\begin{itemize}
\item Traces are translated into \emph{variables}.
\item The moved expression introduces an \emph{abstraction variable} into the LF, which triggers a special translation rule, \emph{predicate abstraction}.
\end{itemize}

\textbf{Predicate abstraction} works as follows:

\begin{itemize}
\item \(\gamma\) is a syntax tree whose only two subtrees are \(x\) and \(\beta\), where \(x\) is an abstraction variable.
\item \(\beta\) is translated as \(\beta'\), an expression of type \(T\)
\item Then translate \(\gamma\) as \(\lambda x\,.\,\beta'\)
\end{itemize}

Now, the exercise proper:

\begin{itemize}
\item Draw synax trees for the sentence (\ref{orga7cf4ec}),  both before and after quantifier raising (assume that ``a European language'' undergoes QR). The syntax tree post-quantifier raising is called the ``LF''.
\item Provide a translation for the sentence into the lambda calculus by compositionally translating each component part, and reducing the result using the reduction rules we've discussed.
\end{itemize}
\end{document}